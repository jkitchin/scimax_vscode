\documentclass[12pt,letterpaper]{article}

\usepackage[T1]{fontenc}

% Hyperlinks
\usepackage[linktocpage,
  pdfstartview=FitH,
  colorlinks,
  linkcolor=blue,
  anchorcolor=blue,
  citecolor=blue,
  filecolor=blue,
  menucolor=blue,
  urlcolor=blue]{hyperref}

% Code highlighting with minted
\usepackage{minted}

% Better tables
\usepackage{booktabs}

% Citations
\usepackage[numbers,super,sort&compress]{natbib}

% User preamble
\usepackage[utf8]{inputenc}
\usepackage{lmodern}
\usepackage{graphicx}
\usepackage{longtable}
\usepackage{float}
\usepackage{wrapfig}
\usepackage{rotating}
\usepackage[normalem]{ulem}
\usepackage{amsmath}
\usepackage{amssymb}
\usepackage{capt-of}
\usepackage{mhchem}

\title{LaTeX Navigation and Structure}
\author{Scimax VS Code Team}
\date{2026-01-19}

\begin{document}

\maketitle

\tableofcontents
\newpage

\section{✅ LaTeX Navigation and Structure}
\label{sec-latex-navigation-and-structure}
CLOSED: [2026-01-19 Mon 09:24]


Scimax VS Code provides comprehensive navigation, structure editing, and tooltips for LaTeX documents. These features mirror the org-mode heading navigation and structure editing commands, adapted for LaTeX section commands and environments.

\subsection{✅ Overview}
\label{sec-overview}
CLOSED: [2026-01-19 Mon 09:24]


LaTeX support in Scimax VS Code includes:


\begin{enumerate}
\item \textbf{Document Outline} - Hierarchical view of sections, environments, and labels

\item \textbf{Section Navigation} - Move between sections at any level

\item \textbf{Structure Editing} - Promote/demote, move, and manipulate sections

\item \textbf{Environment Commands} - Select, change, wrap, and unwrap environments

\item \textbf{Hover Tooltips} - Information about references, citations, packages, and commands

\item \textbf{Speed Commands} - Single-key shortcuts at section start

\item \textbf{Folding} - Collapse and expand sections and environments

\item \textbf{Go to Definition} - Jump to labels, citations, commands, and included files

\item \textbf{Find References} - Find all uses of a label or citation key

\item \textbf{Auto-Completion} - Smart completions for labels, citations, environments, packages

\item \textbf{Multi-File Projects} - Full support for projects using \textbackslash{}input/\textbackslash{}include

\item \textbf{ChkTeX Linting} - Real-time warnings in Problems panel

\item \textbf{Error Navigation} - Cycle through compilation errors

\item \textbf{Rename Symbol} - Rename labels and update all references automatically

\item \textbf{Reference Validation} - Warn about undefined and unused labels

\item \textbf{Inverse SyncTeX} - Click in PDF to jump to source

\item \textbf{Document Formatting} - Format with latexindent

\item \textbf{LaTeX-Aware Spell Checking} - Skip commands and math in spell check

\end{enumerate}


See also:


\begin{itemize}
\item \href{file:21-navigation.org}{Navigation} for general navigation features

\item \href{file:24-keybindings.org}{Keybindings} for a complete reference of keyboard shortcuts

\end{itemize}


\section{✅ Document Outline}
\label{sec-document-outline}
CLOSED: [2026-01-19 Mon 09:24]


The outline view (VS Code's built-in outline panel \texttt{Cmd/C-Shift-o}) shows the structure of your LaTeX document:


\begin{itemize}
\item \textbf{Sections} (part, chapter, section, subsection, etc.)

\item \textbf{Labeled Environments} (figures, tables, equations with labels)

\item \textbf{Labels} for cross-referencing

\end{itemize}


The outline uses hierarchical nesting to show the document structure. Click any item to jump to it.

\subsection{✅ Section Hierarchy}
\label{sec-section-hierarchy}
CLOSED: [2026-01-19 Mon 09:26]


LaTeX sections are organized hierarchically:


\begin{tabular}{lll}
\toprule
Level & Command & Symbol Kind \\
\midrule
0 & \textbackslash{}part & Module \\
1 & \textbackslash{}chapter & Class \\
2 & \textbackslash{}section & Method \\
3 & \textbackslash{}subsection & Function \\
4 & \textbackslash{}subsubsection & Field \\
5 & \textbackslash{}paragraph & Property \\
6 & \textbackslash{}subparagraph & Property \\
\bottomrule
\end{tabular}


Starred variants (e.g., \texttt{\textbackslash{}section*\{...\}}) are also recognized and marked in the outline.


\section{✅ Section Navigation}
\label{sec-section-navigation}
CLOSED: [2026-01-19 Mon 09:26]


Navigate between sections in your LaTeX document using these commands:


\begin{tabular}{lll}
\toprule
Command & Key Binding & Description \\
\midrule
scimax.latex.nextSection & C-c C-n & Jump to next section \\
scimax.latex.previousSection & C-c C-p & Jump to previous section \\
scimax.latex.parentSection & C-c C-u & Jump to parent section \\
scimax.latex.nextSiblingSection & C-c C-f & Jump to next sibling section \\
scimax.latex.previousSiblingSection & C-c C-b & Jump to previous sibling section \\
scimax.latex.firstSection & - & Jump to first section in document \\
scimax.latex.lastSection & - & Jump to last section in document \\
scimax.latex.jumpToSection & C-c C-j / C-c j & Quick pick to jump to any section \\
\bottomrule
\end{tabular}

\subsection{✅ Next/Previous Section}
\label{sec-next-previous-section}
CLOSED: [2026-01-19 Mon 09:26]


Move through all sections in document order:


\begin{minted}{latex}
\section{Introduction}
Content here...

\subsection{Background}    % <- Current position

\subsection{Motivation}    % <- Next section

\section{Methods}          % <- Another next section
\end{minted}

\texttt{nextSection} moves forward through sections regardless of level. \texttt{previousSection} moves backward.

\subsection{✅ Sibling Navigation}
\label{sec-sibling-navigation}
CLOSED: [2026-01-19 Mon 09:26]


Move between sections at the same level:


\begin{minted}{latex}
\section{Introduction}

\subsection{Background}    % <- Current position
\subsection{Motivation}    % <- Next sibling (same level)
\subsection{Goals}         % <- Another sibling

\section{Methods}          % <- NOT a sibling (different level)
\end{minted}

\texttt{nextSiblingSection} moves to the next section at the same hierarchical level.
\texttt{previousSiblingSection} moves backward.

\subsection{✅ Parent Section}
\label{sec-parent-section}
CLOSED: [2026-01-19 Mon 09:26]


Jump to the containing (parent) section:


\begin{minted}{latex}
\section{Methods}            % <- Jump here (parent)
\subsection{Data Collection}
\subsubsection{Sampling}     % <- From here
\end{minted}
\subsection{✅ Jump to Section}
\label{sec-jump-to-section}
CLOSED: [2026-01-19 Mon 18:21]


The \texttt{jumpToSection} command (C-c C-j or C-c j) opens a quick pick showing all sections with:


\begin{itemize}
\item Hierarchical indentation showing structure

\item Section type (\texttt{\textbackslash{}section}, \texttt{\textbackslash{}subsection}, etc.)

\item Line numbers

\end{itemize}


Type to filter, then select a section to jump to it.


\section{✅ Structure Editing}
\label{sec-structure-editing}
CLOSED: [2026-01-19 Mon 09:27]


Edit the document structure by promoting, demoting, and moving sections.


\begin{tabular}{lll}
\toprule
Command & Key Binding & Description \\
\midrule
scimax.latex.promoteSection & Alt+Left & Promote section (e.g., sub→sec) \\
scimax.latex.demoteSection & Alt+Right & Demote section (e.g., sec→sub) \\
scimax.latex.promoteSubtree & Alt+Shift+Left & Promote section and children \\
scimax.latex.demoteSubtree & Alt+Shift+Right & Demote section and children \\
scimax.latex.moveSectionUp & Alt+Up & Move section above sibling \\
scimax.latex.moveSectionDown & Alt+Down & Move section below sibling \\
scimax.latex.markSection & C-c @ & Select entire section \\
scimax.latex.killSection & - & Delete section (copy to clip) \\
scimax.latex.cloneSection & - & Duplicate section \\
scimax.latex.insertSection & - & Insert new section at same level \\
scimax.latex.insertSubsection & - & Insert subsection \\
\bottomrule
\end{tabular}

\subsection{✅ Promote/Demote}
\label{sec-promote-demote}
CLOSED: [2026-01-19 Mon 09:27]


Change the level of a section command:


\begin{minted}{latex}
% Before promote:
\subsection{Methods}

% After promote (Alt+Left):
\section{Methods}

% Before demote:
\section{Results}

% After demote (Alt+Right):
\subsection{Results}
\end{minted}
\subsection{✅ Promote/Demote Subtree}
\label{sec-promote-demote-subtree}
CLOSED: [2026-01-19 Mon 09:27]


Promote or demote a section and all its children:


\begin{minted}{latex}
% Before promoteSubtree on "Methods":
\subsection{Methods}
\subsubsection{Data}
\subsubsection{Analysis}

% After promoteSubtree (Alt+Shift+Left):
\section{Methods}
\subsection{Data}
\subsection{Analysis}
\end{minted}
\subsection{✅ Move Section Up/Down}
\label{sec-move-section-up-down}
CLOSED: [2026-01-19 Mon 09:27]


Swap a section with its previous or next sibling:


\begin{minted}{latex}
% Before moveSectionDown on "Introduction":
\section{Introduction}
Content A...

\section{Methods}
Content B...

% After moveSectionDown (Alt+Down):
\section{Methods}
Content B...

\section{Introduction}
Content A...
\end{minted}

The entire section subtree (including all nested content) is moved.


\section{✅ Environment Commands}
\label{sec-environment-commands}
CLOSED: [2026-01-19 Mon 09:27]


Work with LaTeX environments (\texttt{\textbackslash{}begin\{...\}} / \texttt{\textbackslash{}end\{...\}}):


\begin{tabular}{lll}
\toprule
Command & Key Binding & Description \\
\midrule
scimax.latex.selectEnvironment & C-c C-e & Select entire environment \\
scimax.latex.selectEnvironmentContent & - & Select content (not begin/end) \\
scimax.latex.changeEnvironment & C-c e & Change environment type \\
scimax.latex.wrapInEnvironment & C-c w & Wrap selection in environment \\
scimax.latex.unwrapEnvironment & - & Remove begin/end, keep content \\
scimax.latex.deleteEnvironment & - & Delete entire environment \\
scimax.latex.toggleEnvironmentStar & C-c * & Toggle starred variant \\
scimax.latex.addLabel & C-c C-l & Add \textbackslash{}label to environment \\
scimax.latex.addCaption & - & Add \textbackslash{}caption to figure/table \\
scimax.latex.jumpToMatchingEnvironment & C-c \% & Jump between \textbackslash{}begin and \textbackslash{}end \\
\bottomrule
\end{tabular}

\subsection{✅ Environment Navigation}
\label{sec-environment-navigation}
CLOSED: [2026-01-19 Mon 09:28]


Navigate between environments:


\begin{tabular}{lll}
\toprule
Command & Key Binding & Description \\
\midrule
scimax.latex.nextEnvironment & - & Jump to next environment \\
scimax.latex.previousEnvironment & - & Jump to previous environment \\
scimax.latex.jumpToEnvironment & - & Quick pick all environments \\
\bottomrule
\end{tabular}

\subsection{✅ Change Environment}
\label{sec-change-environment}
CLOSED: [2026-01-19 Mon 09:28]


The \texttt{changeEnvironment} command (C-c e) prompts to change the environment type:


\begin{minted}{latex}
% Before:
\begin{equation}
  E = mc^2
\end{equation}

% After changing to "align":
\begin{align}
  E = mc^2
\end{align}
\end{minted}

Both \texttt{\textbackslash{}begin} and \texttt{\textbackslash{}end} are updated simultaneously.

\subsection{✅ Wrap in Environment}
\label{sec-wrap-in-environment}
CLOSED: [2026-01-19 Mon 09:28]


Select text and use \texttt{wrapInEnvironment} (C-c w) to surround it:


\begin{minted}{latex}
% Before (with "E = mc^2" selected):
E = mc^2

% After wrapping in "equation":
\begin{equation}
  E = mc^2
\end{equation}
\end{minted}
\subsection{✅ Toggle Star}
\label{sec-toggle-star}
CLOSED: [2026-01-19 Mon 09:28]


Toggle the starred variant of an environment:


\begin{minted}{latex}
% Before:
\begin{equation}
  E = mc^2
\end{equation}

% After toggleEnvironmentStar (C-c *):
\begin{equation*}
  E = mc^2
\end{equation*}
\end{minted}

\section{✅ Hover Tooltips}
\label{sec-hover-tooltips}
CLOSED: [2026-01-19 Mon 09:29]


Hover over LaTeX elements to see helpful information:


\begin{tabular}{ll}
\toprule
Element & Information Shown \\
\midrule
\texttt{\textbackslash{}ref\{label\}} & What the label refers to, line number \\
\texttt{\textbackslash{}cite\{key\}} & Citation keys (bib integration planned) \\
\texttt{\textbackslash{}label\{name\}} & Where label is referenced, context \\
\texttt{\textbackslash{}usepackage\{pkg\}} & Package description \\
\texttt{\textbackslash{}begin\{env\}} & Environment description \\
\texttt{\textbackslash{}section\{...\}} & Section hierarchy path \\
\texttt{\textbackslash{}alpha}, \texttt{\textbackslash{}beta}, etc. & Unicode symbol and description \\
\texttt{\textbackslash{}textbf}, etc. & Command description \\
\texttt{\textbackslash{}input\{file\}} & File path, size, preview \\
\bottomrule
\end{tabular}

\subsection{✅ Reference Hover}
\label{sec-reference-hover}
CLOSED: [2026-01-19 Mon 09:29]


Hovering over \texttt{\textbackslash{}ref\{fig:example\}} shows:


\begin{itemize}
\item The label name

\item Line number where label is defined

\item Context (section or environment containing the label)

\item Preview of the labeled line

\end{itemize}

\subsection{✅ Citation Hover}
\label{sec-citation-hover}
CLOSED: [2026-01-19 Mon 09:29]


Hovering over \texttt{\textbackslash{}cite\{key\}} shows rich BibTeX information:


\begin{itemize}
\item Full title

\item Authors with proper formatting

\item Year

\item Journal or book title

\item Volume, number, pages

\item DOI (as clickable link)

\item Abstract preview

\end{itemize}


Supports all natbib and biblatex citation commands:


\begin{itemize}
\item \texttt{\textbackslash{}cite}, \texttt{\textbackslash{}citep}, \texttt{\textbackslash{}citet}

\item \texttt{\textbackslash{}citealt}, \texttt{\textbackslash{}citealp}

\item \texttt{\textbackslash{}citeauthor}, \texttt{\textbackslash{}citeyear}

\item \texttt{\textbackslash{}nocite}

\end{itemize}

\subsection{✅ Package Hover}
\label{sec-package-hover}
CLOSED: [2026-01-19 Mon 09:29]


Hovering over \texttt{\textbackslash{}usepackage\{amsmath\}} shows:


\begin{itemize}
\item Package name

\item Description of what the package provides

\end{itemize}

\subsection{✅ Math Symbol Hover}
\label{sec-math-symbol-hover}
CLOSED: [2026-01-19 Mon 09:29]


Hovering over math commands like \texttt{\textbackslash{}alpha} shows:


\begin{itemize}
\item Unicode representation: α

\item Unicode code point: U+03B1

\item Description: Greek lowercase alpha

\end{itemize}


\section{✅ Speed Commands}
\label{sec-speed-commands}
CLOSED: [2026-01-19 Mon 09:29]


At the beginning of a section line, single keys trigger commands:


\begin{tabular}{llll}
\toprule
Key & Action & Key & Action \\
\midrule
n & Next section & < & Promote section \\
p & Previous section & > & Demote section \\
f & Next sibling & U & Move section up \\
b & Previous sibling & D & Move section down \\
u & Parent section & m & Mark section \\
j & Jump to section & k & Kill section \\
J & Jump to label & c & Clone section \\
g & First section & i & Insert section \\
G & Last section & I & Insert subsection \\
Tab & Toggle fold & N & Narrow to section \\
S-Tab & Cycle global fold & W & Widen \\
? & Show speed commands &  &  \\
\bottomrule
\end{tabular}


Speed commands only activate when cursor is at or before the backslash of a section command.

\subsection{✅ Using Speed Commands}
\label{sec-using-speed-commands}
CLOSED: [2026-01-19 Mon 09:29]


\begin{enumerate}
\item Place cursor at the start of a section line (column 0 or before \texttt{\textbackslash{}})

\item Press a single key

\item The corresponding action executes

\end{enumerate}


Example:


\begin{minted}{latex}
\section{Introduction}   % <- Place cursor here, press 'n' to go to next section
  ^
  |__ Cursor position
\end{minted}

\section{✅ Folding}
\label{sec-folding}
CLOSED: [2026-01-19 Mon 09:29]


LaTeX documents support folding for sections and environments.


\begin{tabular}{lll}
\toprule
Command & Key Binding & Description \\
\midrule
scimax.org.toggleFold & Tab & Toggle fold at cursor \\
scimax.org.cycleGlobalFold & Shift+Tab & Cycle all folds \\
\bottomrule
\end{tabular}

\subsection{✅ Section Folding}
\label{sec-section-folding}
CLOSED: [2026-01-19 Mon 09:29]


Sections fold from the section command to the next section at the same or higher level:


\begin{minted}{latex}
\section{Introduction}...  % <- Folds from here
Content...
\subsection{Background}    % <- Included in fold
More content...
\section{Methods}          % <- Fold ends before this
\end{minted}
\subsection{✅ Environment Folding}
\label{sec-environment-folding}
CLOSED: [2026-01-19 Mon 09:29]


Environments fold from \texttt{\textbackslash{}begin} to \texttt{\textbackslash{}end}:


\begin{minted}{latex}
\begin{figure}...          % <- Folds from here
  \includegraphics{...}
  \caption{...}
\end{figure}               % <- Fold ends here
\end{minted}
\subsection{✅ Global Fold Cycling}
\label{sec-global-fold-cycling}
CLOSED: [2026-01-19 Mon 09:29]


\texttt{Shift+Tab} cycles through three states:


\begin{enumerate}
\item \textbf{OVERVIEW} - All sections folded

\item \textbf{CONTENTS} - Top-level sections visible

\item \textbf{SHOWALL} - Everything expanded

\end{enumerate}


\section{✅ Label Navigation}
\label{sec-label-navigation}
CLOSED: [2026-01-19 Mon 09:31]


Navigate to labels in your document:


\begin{tabular}{lll}
\toprule
Command & Key Binding & Description \\
\midrule
scimax.latex.jumpToLabel & C-c l & Quick pick all labels \\
\bottomrule
\end{tabular}


The quick pick shows:


\begin{itemize}
\item Label name

\item Context (section or environment)

\item Line number

\end{itemize}


\section{✅ Compile and Preview}
\label{sec-compile-and-preview}
CLOSED: [2026-01-19 Mon 09:31]


Compile LaTeX documents and view the generated PDF.


\begin{tabular}{lll}
\toprule
Command & Key Binding & Description \\
\midrule
scimax.latex.compile & C-c C-c & Compile the document \\
scimax.latex.viewPdf & C-c C-v & View PDF (external viewer) \\
scimax.latex.viewPdfPanel & C-c v & View PDF (built-in panel) \\
scimax.latex.compileAndView & C-c C-a & Compile and view PDF \\
scimax.latex.clean & C-c C-k & Clean auxiliary files \\
scimax.latex.syncTexForward & C-c g & SyncTeX forward search \\
scimax.latex.wordCount & - & Count words (approximate) \\
\bottomrule
\end{tabular}

\subsection{✅ Compilation}
\label{sec-compilation}
CLOSED: [2026-01-19 Mon 09:31]


The \texttt{compile} command runs the configured LaTeX compiler (default: pdflatex):


\begin{enumerate}
\item Saves the document

\item Compiles with \texttt{-interaction}nonstopmode= for error handling

\item Shows compilation errors with line numbers

\item Outputs detailed log to the "LaTeX" output channel

\end{enumerate}


Configure the compiler via \texttt{scimax.latex.compiler} setting:


\begin{itemize}
\item \texttt{pdflatex} (default)

\item \texttt{xelatex}

\item \texttt{lualatex}

\item \texttt{latexmk}

\end{itemize}

\subsection{✅ SyncTeX Support}
\label{sec-synctex-support}
CLOSED: [2026-01-19 Mon 09:31]


SyncTeX allows jumping between source and PDF positions.


Supported PDF viewers:


\begin{itemize}
\item \textbf{macOS}: Skim (recommended)

\item \textbf{Linux}: Zathura

\item \textbf{Windows}: SumatraPDF

\end{itemize}


Configure via \texttt{scimax.latex.pdfViewer} setting.

\subsection{⚠️ Built-in PDF Viewer Panel}
\label{sec-built-in-pdf-viewer-panel}
For an Overleaf-like experience, use the built-in PDF viewer panel (\texttt{C-c v}):


\begin{itemize}
\item Opens PDF in a VS Code panel beside your source

\item Auto-refreshes when you recompile

\item Zoom controls (+, -, Fit)

\item Page navigation

\item Bidirectional sync with text-based highlighting

\item No external PDF viewer required

\end{itemize}


This is ideal for side-by-side editing without leaving VS Code.


\begin{tabular}{lll}
\toprule
Viewer Option & Key & Description \\
\midrule
Built-in panel & C-c v & PDF in VS Code panel (Overleaf-like) \\
External viewer & C-c C-v & Opens in Skim/Zathura/SumatraPDF \\
\bottomrule
\end{tabular}

\subsubsection{Bidirectional Sync}
\label{sec-bidirectional-sync}
The built-in PDF viewer supports precise bidirectional synchronization:


to that location in the PDF. The word is highlighted with a yellow flash.


to the corresponding line in your LaTeX source.


The forward sync uses SyncTeX coordinates combined with text-layer search
to find the exact word you clicked, even if SyncTeX coordinates are imprecise.

\subsubsection{Debug Popup Setting}
\label{sec-debug-popup-setting}
To see SyncTeX coordinates and debug information when syncing:


\begin{minted}{json}
{
  "scimax.latex.showSyncDebugPopup": true
}
\end{minted}

Default is \texttt{false}. When enabled, shows a popup with page, coordinates,
source line/column, and search text for troubleshooting sync issues.

\subsection{⚠️ Insert Commands}
\label{sec-insert-commands}
Quick insertion of common LaTeX elements:


\begin{tabular}{ll}
\toprule
Command & Description \\
\midrule
scimax.latex.insertFigure & Insert figure environment snippet \\
scimax.latex.insertTable & Insert table with booktabs snippet \\
scimax.latex.insertEquation & Insert equation environment \\
\bottomrule
\end{tabular}


These commands insert snippets with tab stops for easy navigation.


\section{⚠️ Go to Definition}
\label{sec-go-to-definition}
Jump to the definition of labels, citations, commands, and files:


\begin{tabular}{ll}
\toprule
Trigger & What it does \\
\midrule
\texttt{\textbackslash{}ref\{label\}} & Jump to the \texttt{\textbackslash{}label\{label\}} definition \\
\texttt{\textbackslash{}cite\{key\}} & Jump to the BibTeX entry in the .bib file \\
\texttt{\textbackslash{}input\{file\}} & Open the included file \\
\texttt{\textbackslash{}mycommand} & Jump to \texttt{\textbackslash{}newcommand} or \texttt{\textbackslash{}def} definition \\
\bottomrule
\end{tabular}


Use \texttt{F12} or \texttt{Ctrl+Click} (standard VS Code "Go to Definition").


Supports cross-file navigation in multi-file projects.


\section{✅ Find All References}
\label{sec-find-all-references}
CLOSED: [2026-01-19 Mon 09:55]


Find all usages of a label or citation:


\begin{tabular}{ll}
\toprule
Position on & Shows all \\
\midrule
\texttt{\textbackslash{}label\{name\}} & All \texttt{\textbackslash{}ref\{name\}} references \\
\texttt{\textbackslash{}cite\{key\}} & All citations of that key \\
\bottomrule
\end{tabular}


Use \texttt{Shift+F12} or right-click → "Find All References".


Works across all files in the project.


\section{✅ Auto-Completion}
\label{sec-auto-completion}
CLOSED: [2026-01-19 Mon 09:56]


Smart completions trigger as you type:


\begin{tabular}{ll}
\toprule
Context & Completions \\
\midrule
\texttt{\textbackslash{}ref\{} & All labels in project with context \\
\texttt{\textbackslash{}cite\{} & All BibTeX keys with author/year/title \\
\texttt{\textbackslash{}begin\{} & Common environments (equation, figure, etc.) \\
\texttt{\textbackslash{}usepackage\{} & Common packages with descriptions \\
\texttt{\textbackslash{}includegraphics\{} & Image files (.png, .jpg, .pdf, etc.) \\
\texttt{\textbackslash{}input\{} & .tex files in directory \\
\bottomrule
\end{tabular}


Completions show:


\begin{itemize}
\item Label completions include file name and line context

\item Citation completions show author, year, and title

\item Environment completions include descriptions

\item Package completions describe what each package provides

\end{itemize}


\section{✅ Multi-File Project Support}
\label{sec-multi-file-project-support}
CLOSED: [2026-01-19 Mon 09:56]


Projects with multiple files (using \texttt{\textbackslash{}input} or \texttt{\textbackslash{}include}) are fully supported:


\begin{itemize}
\item \textbf{Master document detection}: Automatically finds the main .tex file

\item \textbf{Cross-file labels}: Complete and navigate labels defined in any project file

\item \textbf{Shared bibliography}: Finds .bib files referenced anywhere in project

\item \textbf{Project-wide search}: Find References searches all included files

\end{itemize}


The master document is detected by looking for \texttt{\textbackslash{}documentclass}.

\subsection{✅ Project Structure}
\label{sec-project-structure}
CLOSED: [2026-01-19 Mon 09:56]


\begin{minted}{text}
thesis/
├── main.tex              # Master document (has \documentclass)
│   └── \input{chapters/intro}
│   └── \input{chapters/methods}
│   └── \bibliography{refs}
├── chapters/
│   ├── intro.tex        # \label{sec:intro} defined here
│   └── methods.tex      # \ref{sec:intro} works here too
└── refs.bib              # Bibliography
\end{minted}

\section{✅ ChkTeX Linting}
\label{sec-chktex-linting}
CLOSED: [2026-01-19 Mon 09:56]


Real-time linting with ChkTeX catches common LaTeX issues:


\begin{itemize}
\item Missing or extra braces

\item Spacing issues

\item Style recommendations

\end{itemize}


Errors appear in the Problems panel and as squiggles in the editor.

\subsection{✅ Configuration}
\label{sec-configuration}
CLOSED: [2026-01-19 Mon 09:56]


\begin{minted}{json}
{
  "scimax.latex.enableChktex": true  // Enable/disable linting (default: true)
}
\end{minted}

Requires \texttt{chktex} to be installed:


\begin{itemize}
\item \textbf{macOS}: \texttt{brew install chktex}

\item \textbf{Linux}: \texttt{apt install chktex}

\item \textbf{Windows}: Included with MiKTeX/TeX Live

\end{itemize}


\section{⚠️ Error Navigation}
\label{sec-error-navigation}
Navigate through compilation errors:


\begin{tabular}{lll}
\toprule
Command & Key Binding & Description \\
\midrule
scimax.latex.nextError & C-c ` & Jump to next error \\
scimax.latex.previousError & C-c Shift+` & Jump to previous error \\
scimax.latex.showErrors & C-c C-e & Show all errors in quick pick \\
\bottomrule
\end{tabular}


Errors are collected after compilation and persist until the next build.

\subsection{✅ Error Display}
\label{sec-error-display}
CLOSED: [2026-01-19 Mon 09:57]


When compilation fails:


\begin{enumerate}
\item First error is shown in a notification with "Go to Error" button

\item The Problems panel shows all parsed errors

\item Use navigation commands to cycle through errors

\end{enumerate}


Errors include:


\begin{itemize}
\item File path (supports multi-file projects)

\item Line number

\item Error message

\end{itemize}


\section{✅ Hover Previews}
\label{sec-hover-previews}
CLOSED: [2026-01-19 Mon 09:18]


Hover over LaTeX elements to see previews and information.

\subsection{✅ Equation Preview}
\label{sec-equation-preview}
CLOSED: [2026-01-19 Mon 09:18]


Hover over any math expression to see a rendered preview:


\begin{itemize}
\item Inline math: \texttt{\$E}mc\textasciicircum{}2\$=, \texttt{\textbackslash{}(E}mc\textasciicircum{}2\textbackslash{})=

\item Display math: \texttt{\$\$...\$\$\$}, \texttt{\textbackslash{}[...\textbackslash{}]}

\item Environments: \texttt{equation}, \texttt{align}, \texttt{gather}, etc.

\end{itemize}


The preview:


\begin{itemize}
\item Renders using the system LaTeX installation

\item Respects document preamble (packages, custom commands)

\item Adapts to dark/light VS Code theme

\item Shows the source LaTeX below the preview

\end{itemize}

\subsection{✅ Figure Preview}
\label{sec-figure-preview}
CLOSED: [2026-01-19 Mon 09:18]


Hover over \texttt{\textbackslash{}includegraphics\{...\}} to see:


\begin{itemize}
\item Image preview (PNG, JPG, SVG, GIF, WebP)

\item File path and size

\item Last modified date

\end{itemize}


PDF and EPS files show file information without preview.


\section{✅ Rename Symbol}
\label{sec-rename-symbol}
CLOSED: [2026-01-19 Mon 09:17]


Rename a label and automatically update all references across the project:


\begin{enumerate}
\item Place cursor on a \texttt{\textbackslash{}label\{name\}} or \texttt{\textbackslash{}ref\{name\}}

\item Press \texttt{F2} or right-click → "Rename Symbol"

\item Enter the new name

\item All \texttt{\textbackslash{}label\{\}} and \texttt{\textbackslash{}ref\{\}} commands using that name are updated

\end{enumerate}


Works across all files in multi-file projects.

\subsection{✅ Supported Reference Commands}
\label{sec-supported-reference-commands}
CLOSED: [2026-01-19 Mon 09:17]


Rename works with all reference variants:


\begin{itemize}
\item \texttt{\textbackslash{}label\{...\}} - the definition

\item \texttt{\textbackslash{}ref\{...\}}, \texttt{\textbackslash{}eqref\{...\}}, \texttt{\textbackslash{}pageref\{...\}}

\item \texttt{\textbackslash{}autoref\{...\}}, \texttt{\textbackslash{}cref\{...\}}, \texttt{\textbackslash{}Cref\{...\}}

\end{itemize}


\section{✅ Reference Validation}
\label{sec-reference-validation}
CLOSED: [2026-01-19 Mon 09:17]


Automatic validation of label references:


\begin{tabular}{lll}
\toprule
Issue & Severity & Description \\
\midrule
Undefined reference & Error & \texttt{\textbackslash{}ref\{name\}} but no \texttt{\textbackslash{}label\{name\}} \\
Unused label & Warning & \texttt{\textbackslash{}label\{name\}} with no \texttt{\textbackslash{}ref\{name\}} \\
\bottomrule
\end{tabular}


Diagnostics appear in the Problems panel and as squiggles in the editor.

\subsection{✅ Configuration}
\label{sec-configuration}
CLOSED: [2026-01-19 Mon 09:17]


\begin{minted}{json}
{
  "scimax.latex.validateReferences": true  // Enable/disable (default: true)
}
\end{minted}

\section{✅ Inverse SyncTeX}
\label{sec-inverse-synctex}
CLOSED: [2026-01-19 Mon 09:17]


Click in PDF to jump back to the corresponding source location.

\subsection{👀 Setup}
\label{sec-setup}
Run the command \texttt{Scimax LaTeX: Show Inverse SyncTeX Command} to get the
configuration command for your PDF viewer.


\begin{tabular}{ll}
\toprule
PDF Viewer & Configuration \\
\midrule
Skim & Preferences → Sync → Preset: Custom; Command: \texttt{code} \\
SumatraPDF & Settings → Options → Set inverse search command \\
Zathura & Set \texttt{synctex-editor-command} in config \\
\bottomrule
\end{tabular}

\subsection{✅ URI Handler}
\label{sec-uri-handler}
CLOSED: [2026-01-19 Mon 09:17]


The extension also registers a URI handler for \texttt{vscode://} URLs:


\begin{minted}{text}
vscode://file/path/to/file.tex:42
\end{minted}

This allows PDF viewers to communicate directly with VS Code.


\section{✅ Document Formatting}
\label{sec-document-formatting}
CLOSED: [2026-01-19 Mon 09:17]


Format LaTeX documents using \texttt{latexindent}:


\begin{tabular}{lll}
\toprule
Command & Key Binding & Description \\
\midrule
scimax.latex.format & C-c C-q & Format entire document \\
\bottomrule
\end{tabular}

\subsection{✅ Format on Save}
\label{sec-format-on-save}
CLOSED: [2026-01-19 Mon 09:12]


Enable automatic formatting when saving:


\begin{minted}{json}
{
  "scimax.latex.formatOnSave": true,          // Auto-format on save
  "scimax.latex.latexindentPath": "latexindent"  // Path to latexindent
}
\end{minted}
\subsection{✅ Requirements}
\label{sec-requirements}
CLOSED: [2026-01-19 Mon 09:12]


\texttt{latexindent} must be installed:


\begin{itemize}
\item \textbf{macOS}: Included with MacTeX, or \texttt{brew install latexindent}

\item \textbf{Linux}: \texttt{apt install texlive-extra-utils} or \texttt{cpanm YAML::Tiny File::HomeDir Unicode::GCString}

\item \textbf{Windows}: Included with MiKTeX/TeX Live

\end{itemize}

\subsection{✅ Customization}
\label{sec-customization}
CLOSED: [2026-01-19 Mon 09:12]


Create a \texttt{.latexindent.yaml} file in your project to customize formatting rules.
See the \href{[[https://latexindentpl.readthedocs.io/][latexindent documentation]]}{latexindent documentation} for options.


\section{✅ LaTeX-Aware Spell Checking}
\label{sec-latex-aware-spell-checking}
CLOSED: [2026-01-19 Mon 09:12]


The extension provides infrastructure for LaTeX-aware spell checking that
skips commands, math, and comments.

\subsection{✅ Regions Excluded from Spell Check}
\label{sec-regions-excluded-from-spell-check}
CLOSED: [2026-01-19 Mon 09:12]


\begin{tabular}{ll}
\toprule
Pattern & Example \\
\midrule
Commands with arguments & \texttt{\textbackslash{}textbf\{word\}} \\
Inline math & \texttt{\$x\textasciicircum{}2 + y\textasciicircum{}2\$} \\
Display math & \texttt{\$\$..\$\$} and \texttt{\textbackslash{}[..\textbackslash{}]} \\
Environments & \texttt{\textbackslash{}begin\{...\}...\textbackslash{}end\{...\}} \\
Comments & \texttt{\% comment text} \\
\bottomrule
\end{tabular}

\subsection{✅ User Dictionary}
\label{sec-user-dictionary}
CLOSED: [2026-01-19 Mon 09:12]


Add words to your personal dictionary:


\begin{tabular}{ll}
\toprule
Command & Description \\
\midrule
scimax.latex.addToDictionary & Add word at cursor to dictionary \\
\bottomrule
\end{tabular}


The dictionary is stored per-workspace and persists across sessions.


\section{✅ Related Topics}
\label{sec-related-topics}
CLOSED: [2026-01-19 Mon 09:12]


\begin{itemize}
\item \href{file:11-latex-preview.org}{LaTeX Preview} - Live PDF preview with SyncTeX

\item \href{file:10-export.org}{Export} - Export org to LaTeX

\item \href{file:19-references.org}{References} - BibTeX bibliography management

\item \href{file:23-speed-commands.org}{Speed Commands} - Org-mode speed commands

\item \href{file:24-keybindings.org}{Keybindings} - Complete keybinding reference

\end{itemize}


\end{document}